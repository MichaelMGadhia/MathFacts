\documentclass{article}

\usepackage{amsmath, amssymb, amsthm}

% YYYY-MM-DD Date format
\usepackage[style=iso]{datetime2}

\newcommand{\N}{\mathbb{N}}
\newcommand{\R}{\mathbb{R}}

\title{More Real Than Natural: Introducing Diagonalization And Countability}
\author{Michael Gadhia}
\date{\today}

\begin{document}
\maketitle

% Introduction, natural numbers and real numbers
\section{Number Sets}
You are probably familiar with these sets of numbers, even if the names are new.
We won't worry about the definition of a set for now, so think of these as collections or categories of numbers.
The \emph{natural numbers} are the numbers $0, 1, 2, 3...$ and so on, so zero and all of the positive whole numbers (positive integers).
We write the set of natural numbers as $\N$.
How many natural numbers are there?
I hope it's clear that there are infinitely many, as adding one to any natural number is still a natural number.
So the size of $\N$, the number of things in it, is infinite.
But now we will introduce a set of numbers that's larger.
Yes, larger than this infinitely large set.

The \emph{real numbers} are all decimal numbers: all the natural numbers and negatives, but also decimals like one half, and irrational numbers like $\pi$ and $\sqrt{2}$.
We write the set of real numbers as $\R$.
There are of course infinitely many of these, so the size of $\R$ is infinite.
But we will see that $\R$ and $\N$ do not contain the same number of things.
In fact, we will see that there are fewer numbers in $\N$ than there are real numbers between zero and one!

\section{Counting Infinity}

\end{document}