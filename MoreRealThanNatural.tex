\documentclass{article}

\usepackage{amsmath, amssymb, amsthm}

% YYYY-MM-DD Date format
\usepackage[style=iso]{datetime2}

\newcommand{\N}{\mathbb{N}}
\newcommand{\R}{\mathbb{R}}

\title{More Real Than Natural: Introducing Diagonalization And Countability}
\author{Michael Gadhia}
\date{\today}

\begin{document}
\maketitle

% Introduction, natural numbers and real numbers
\section{Number Sets}
You are probably familiar with these sets of numbers, even if the names are new.
We won't worry about the definition of a set for now, so think of these as collections or categories of numbers.
The \emph{natural numbers} are the numbers $0, 1, 2, 3...$ and so on, so zero and all of the positive whole numbers (positive integers).
We write the set of natural numbers as $\N$.
How many natural numbers are there?
I hope it's clear that there are infinitely many, as adding one to any natural number is still a natural number.
So the size of $\N$, the number of things in it, is infinite.
But now we will introduce a set of numbers that's larger.
Yes, larger than this infinitely large set.

The \emph{real numbers} are all decimal numbers: all the natural numbers and negatives, but also decimals like one half, and irrational numbers like $\pi$ and $\sqrt{2}$.
We write the set of real numbers as $\R$.
There are of course infinitely many of these, so the size of $\R$ is infinite.
But we will see that $\R$ and $\N$ do not contain the same number of things.
In fact, we will see that there are fewer numbers in $\N$ than there are real numbers between zero and one!

% Setting up for proof sketch
\section{Counting Infinity}
We will show a sketch of a proof for the fact that there are more real numbers between 0 and 1 than natural numbers.
This is not a rigorous statement of the proof, but will hopefully be clear enough to convey the technique used, and convince a typical reader that this is true.
First, we will explain how we compare the size of two sets: matching things in one set to things in another.
As an example, we know there are just as many positive integers as negative integers, since we can match each number to its negative.
Then, every number in one set matches exactly one number in the other set, so they must be the same size.
This technique also tells us that there are as many natural numbers as positive integers (natura numbers without 0), since every natural number $n$ can be matched with the positive integer $n+1$.
But if adding or removing one element doesn't change the size $\N$, why can something be bigger than it?

We will represent all real numbers between 0 and 1 as decimals, starting with a zero and a decimal point.
This would make the number one third be the infinitely long $0.3333...$ and will include infinitely many trailing zeroes for numbers that terminate, so one half is $0.5000...$.
In this way, every number will have an infinitely long "tail" of digits after the decimal point.
Now, we are ready to outline the proof.

\end{document}