\documentclass{article}

\usepackage{amsmath, amssymb, amsthm}

% YYYY-MM-DD Date format
\usepackage[style=iso]{datetime2}

\newcommand{\N}{\mathbb{N}}
\newcommand{\R}{\mathbb{R}}

\title{More Real Than Natural: Introducing Diagonalization And Countability}
\author{Michael Gadhia}
\date{\today}

\begin{document}
\maketitle

% Introduction, natural numbers and real numbers
\section{Number Sets}
You are probably familiar with these sets of numbers, even if the names are new.
We won't worry about the definition of a set for now, so think of these as collections or categories of numbers.
The \emph{natural numbers} are the numbers $0, 1, 2, 3...$ and so on, so zero and all of the positive whole numbers (positive integers).
We write the set of natural numbers as $\N$.
How many natural numbers are there?
I hope it's clear that there are infinitely many, as adding one to any natural number is still a natural number.
So the size of $\N$, the number of things in it, is infinite.
But now we will introduce a set of numbers that's larger.
Yes, larger than this infinitely large set.

The \emph{real numbers} are all decimal numbers: all the natural numbers and negatives, but also decimals like one half, and irrational numbers like $\pi$ and $\sqrt{2}$.
We write the set of real numbers as $\R$.
There are of course infinitely many of these, so the size of $\R$ is infinite.
But we will see that $\R$ and $\N$ do not contain the same number of things.
In fact, we will see that there are fewer numbers in $\N$ than there are real numbers between zero and one!

% Setting up for proof sketch
\section{Counting Infinity}
We will show a sketch of a proof for the fact that there are more real numbers between 0 and 1 than natural numbers.
This is not a rigorous statement of the proof, but will hopefully be clear enough to convey the technique used, and convince a typical reader that this is true.
First, we will explain how we compare the size of two sets: matching things in one set to things in another.
As an example, we know there are just as many positive integers as negative integers, since we can match each number to its negative.
Then, every number in one set matches exactly one number in the other set, so they must be the same size.
This technique also tells us that there are as many natural numbers as positive integers (natura numbers without 0), since every natural number $n$ can be matched with the positive integer $n+1$.
But if adding or removing one element doesn't change the size $\N$, why can something be bigger than it?

We will represent all real numbers between 0 and 1 as decimals, starting with a zero and a decimal point.
This would make the number one third be the infinitely long $0.3333...$ and will include infinitely many trailing zeroes for numbers that terminate, so one half is $0.5000...$.
In this way, every number will have an infinitely long ``tail" of digits after the decimal point.
Now, we are ready to outline the proof.

% Outline of proof by diagonalization
\section{Diagonalization}
We will use a contradiction: assume that there are just as many natural numbers as real numbers between zero and one, and then show that assuming that led to something impossible.
Let's say the sizes are equal, so you can match every element of each set to exactly one element in the other.
Since we want to prove this is impossible, we will make no other assumptions about how to do that, so this must work for any possible matching of elements.
I will write an example with random numbers to illustrate this, but it will become clear that the numbers could be changed without affecting the result.
Imagine such a matching below:
%
% Adapted from https://tex.stackexchange.com/a/501747
\begin{alignat*}{7}
 0 & = 0. & & 1 & & 4 & & 5 & & 0 & & 9 & & \ldots \\
 1 & = 0. & & 1 & & 2 & & 1 & & 1 & & 1 & & \ldots \\
 2 & = 0. & & 4 & & 2 & & 1 & & 5 & & 3 & & \ldots \\
 3 & = 0. & & 5 & & 1 & & 2 & & 6 & & 2 & & \ldots \\
 4 & = 0. & & 2 & & 4 & & 7 & & 7 & & 5 & & \ldots\\[-1.33ex]
 \vdots
\end{alignat*}
%
If we can prove any matching like this is missing something, we know our assumption was impossible.
We will not only show this now, but also use the matching to give a specific element that's missing!

First, choose two digits, a primary and a fallback.
For the example above, I will use 1 first, and 6 as a fallback.
Now, I will look at the first number in the matching (the one matched with zero), and see that the first digit after the decimal point is a 1.
Since I want to make sure our new number isn't on the list, the first digit will be 6.
Then, I look at the \emph{second} digit of the \emph{second} number on the list (matched with 1), and see that it is a 2.
This means I can put a 1 as the second digit.
Now, I take the third digit of the third number, a 1, and put a 6.
The fourth digit of the fourth number is a 6, so I put a 1.
The fifth digit of the fifth number is a 5, so I put a 1.
The number looks like $0.61611...$, and to get its infinite ``tail'' of digits, I do this for the entire list.

Now, I claim that this new number appears nowhere in the matching, meaning that the matching is missing something.
To show why that's true, let's look at each number one by one to see if it's in the matching.
Is it the first number in the matching?
No, because its first digit is different, a 1 versus a 6.
Is it the second number in the matching?
No, its second digit is different, a 2 versus a 1.
Continuing this, we find that this number is different from every item in the list in at least one digit place.
So, we've found a number that isn't in the matching but should be, and shown that something is missing!

If this number were added to a matching, we could simply redo this process with any two digits to produce another missing number, so this works for all matchings.
This means our assumption that the two sets were the same size led to something impossible, and must therefore be false.
So they are not the same size, and since what's missing is a real number, there are more real numbers between zero and one than there are natural numbers.


% Discussion, mention of formal terms
\section{Discussion}
This is a simple presentation of Cantor's diagonal argument.
This was based on versions I got from several college professors, as this is an engaging, approachable topic often used as an example to illustrate creative proof techniques.
Hopefully a finished version of this document will be easier to follow, more approachable in its language, and have a better visual representation of the technique.


\end{document}