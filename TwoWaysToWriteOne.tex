\documentclass{article}

\usepackage{amsmath, amssymb, amsthm}

% YYYY-MM-DD Date format
\usepackage[style=iso]{datetime2}

% Non-numbered theorem
\newtheorem*{theorem*}{Theorem}

\title{Two Ways To Write One: Unintuitive Repeating Decimals}
\author{Michael Gadhia}
\date{\today}

\begin{document}
\maketitle

% Introduction and presentation of `paradox'
\section{Repeating Decimals}

You probably know what one third looks like as a decimal:
\[
\frac{1}{3} = 0.3333...
\]
The ``$...$'' represents infinitely many threes after the decimal point.
There are many examples of repeating decimals, and they may have more than one digit repeated.
For instance, dividing $1$ by $11$ gives $0.09090909...$ where the pattern ``$09$'' appears infinitely many times.
We sometimes use a bar to indicate repetition of a pattern, so one third is written $0.\overline{3}$ and one eleventh is written $0.\overline{09}$.

Now, consider this repeating decimal: $0.9999...$, which we can also write as $0.\overline{9}$.
When first looking at it, this number appears to be almost equal to one.
But in fact, this number is \emph{exactly} equal to one, and is just a second way of writing it!

This is a document about that fact, simple to write but often difficult to understand and believe:

\[
0.9999... = 1
\]

If the equation above seems unbelievable, or just feels wrong, don't worry.
We'll look at several ways to think about it, from simple arithmetic to a complete formal proof.

% Explaining why this is true
\section{Arguments and Proofs}

We will discuss the fact that $0.\overline{9} = 1$, and ultimately present a mathematical proof for it.
Before that, we'll look at intuitive arguments to get a sense for why it's true.

% Algebraic arguments
\subsection{Intuition}

The first way to think about it may have already occurred to you, based on the first example of a repeating decimal.
One third is $0.\overline{3}$, and multiplying this by three would look like $0.\overline{9}$, but three thirds is exactly one.
While this isn't a complete proof, this argument is often enough to convince students this is true.
There is another way to demonstrate this with grade-school algebra without relying on being given a definition of one third, which is given below.

Let $x = 0.9999...$, and consider what happens when we multiply $x$ by $10$.
Since this can be seen as `moving the decimal point' of a decimal number, we get $10x = 9.9999...$.
Now, we can subtract $x$ from both sides and get the following:
\begin{align*}
    10x - x &= 9.9999... - x\\
    9x &= 9.9999... - (0.9999...)\\
    9x &= 9\\
    x &= 1
\end{align*}


% Formal proofs presented with either approachable/defined terminology or warning about unfamiliarity
\subsection{Rigorous Proof}
%https://en.wikipedia.org/wiki/0.999...#Rigorous_proof
This will require us to define $0.\overline{9}$ more explicitly.
We will define it in terms of an infinite sequence of numbers that approach it, $0.9$, $0.99$, and so on.
We'll name them based on the number of nines: $x_1 = 0.9$, $x_2 = 0.99$, $x_3 = 0.999$, and so on.
This creates a sequence where $x_n$ has $n$ nines after a decimal point.
We then have our definition of $0.\overline{9}$: it is the number this sequence approaches as $n$ gets larger.
The sequence will never reach this number, but every term gets closer to it, this is the \emph{limit} of the sequence.
Now, we will prove that this number that the sequence approaches is in fact equal to 1.

\begin{theorem*}
    The sequence of numbers $x_1, x_2, \ldots$ defined above approaches 1.
    In other words, $0.\overline{9} = 1$.
\end{theorem*}
\begin{proof}
    Subtract 1 from every term $x_1, x_2, \ldots$ in the sequence.
    $1 - 0.9 = 0.1$, $1 - 0.99 = 0.01$, and so on.
    For any $n$, $1 - x_n$ is zeroes followed by a 1 at the $n$th position, equal to $\frac{1}{10^n}$.

    We now want to show that $1 - x_n$ approaches zero, since the sequence approaches $0.\overline{9}$, and showing that $1 - 0.\overline{9}$ proves they are equal.
    So we must show that there is nothing greater than zero that is always smaller than $1 - x_n$ for every $n$.
    This relies on the fact that any number has a whole number bigger than it, which we call the Archimedean property (a fancy name for something very simple).

    If there was a number greater than zero that was always smaller than $1 - x_n$, some number $m$, it would have to be smaller than $\frac{1}{10^n}$ for every value of $n$, since that equals $1 - x_n$.
    But we can apply the Archimedian property: if every number has some whole number $a$ bigger than it, then any number (specifically positive ones in this case) has some (positive) $\frac{1}{a}$ smaller than it.
    So, for $m$, which is a positive number greater than zero, there must be some $\frac{1}{a}$ that's smaller.
    But there must be some $n$ for which $10^n$ is greater than $a$, so $\frac{1}{10^2}$ is smaller than $m$.
    That contradicts our definition of $m$, so $m$ cannot exist!
    From that, we know that $1 - x_n$ approaches zero, so $1 - 0.\overline{9} = 0$, and therefore $0.\overline{9} = 1$.

\end{proof}


\end{document}